%!TEX root = ./ai-jst.tex
%-------------------------------------------------------------------------------
%                            BAB V
%               	 KESIMPULAN dan SARAN
%-------------------------------------------------------------------------------

\chapter{KESIMPULAN DAN SARAN}                

\section{Kesimpulan}
  Jaringan syaraf tiruan merupakan salah satu penerapan yang sangat membantu untuk memecahkan masalah dan persoalan \emph{non-linear} didalam kehidupan sehari-hari, sebagai contoh permasalahan yang menjadi fokus penelitian kami, dalam menilai kelayakan tugas akhir mahasiswa, walaupun metode \emph{Backpropagation} tergolong metode yang sangat dasar dalam dunia jaringan syaraf tiruan tapi metode ini tergolong sangat mampu untuk memcahkan permasalahan yang seperti kami lakukan didalam penelitian ini.Lebih jauh kagi kami menyadari keterbatasan kami dalam melakukan penelitian ini, salah satu nya adalah dengan memanfaatkan \emph{framework} jst dari javascript. 

\section{Saran}
Saran yang dapat ditulis didalam laporan ini adalah:
\begin{enumerate}
	\item Untuk penelitian selajutnya diharapkan kami mampu membangun sebuah model jst tanpa ada bantuan dari \emph{framework} dan dengan \emph{dataset} yang memadai.
	\item Metode yang digunakan tidak hanya \emph{backpropagation} saja, mungkin juga bisa mengunakan metode LSTM untuk data-data yang bersifat runtut waktu. 
\end{enumerate}
% Baris ini digunakan untuk membantu dalam melakukan sitasi
% Karena diapit dengan comment, maka baris ini akan diabaikan
% oleh compiler LaTeX.
\begin{comment}
\bibliography{daftar-pustaka}
\end{comment}
