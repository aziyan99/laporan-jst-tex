%!TEX root = ./ai-jst.tex
%-------------------------------------------------------------------------------
% 								BAB I
% 							LATAR BELAKANG
%-------------------------------------------------------------------------------

\chapter{PENDAHULUAN}

\section{Latar Belakang}
Jaringan syaraf tiruan (JST) atau lebih dikenal dengan \textit{Artificial Neural Network} merupakan sebuah sistem pemrosesan informasi yang memiliki karakteristik kinerja tertentu yang sama dengan jaringan saraf biologis otak manusia \cite{fausett1994fundamentals}. Secara sederhana, jaringan syaraf tiruan adalah sebuah alat pemodelan data statistik non-linier. yang dapat digunakan untuk memodelkan hubungan yang kompleks antara input dan output untuk menemukan pola-pola pada data \cite{wikipedia}. \par
Jaringan syaraf biologis mempunyai tiga komponen utama yaitu: \textit{dendrites, soma} dan \textit{axon}. \textit{dendrites} berkerja sebagai penerima sinyal dari \textit{neuron-neuron}, \textit{soma} berkerja sebagai penjumlah sinyal-sinyal yang diterima, ketika sinyal-sinyal yang diterima dirasa cukup \textit{soma} akan meneruskan sinyal-sinyalnya ke sel-sel yang lain melalui \textit{axon}. Pada bagian \textit{soma} ini, ada kalanya ia meneruskan sinyalnya dan ada kalanya tidak, hal ini bisa dimisalkan dengan proses klasifikasi atau penentuan\cite{fausett1994fundamentals}. Dengan analogi yang demikian model JST dibangun. \par
Penggunaan JST dalam menyelesaikan masalah kian populer, hal ini disebabkan oleh kemampuan JST dalam memodelkan masalah \textit{linear} atau \textit{non-linear} kemudian mempelajari hubungan-hubungan antara \textit{variable-variable} yang diberikan dan pada akhirnya JST mampu memberikan keputusan berdasarkan hasil dari hubungan-hubungan \textit{variable-variable} tadi untuk menyelesaikan permasalahan yang dimaksud. Contoh penerapan JST didalam bidang pendidikan adalah mengklasifikasi tugas akhir mahasiswa apakah masuk dikategori layak atau tidak layak, alih-alih memeriksa satu persatu kelayakan tugas akhir mahasiswa, mengapa tidak melatih sebuah model JST dengan memberikan contoh tugas akhir yang layak dan tugas akhir yang tidak layak, kemudian biarkan JST sendiri mengklarifikasi kelayakan tugas-tugas akhir tersebut. \par 
Arsitektur didalam JST dalam memodelkan jaringan syaraf secara umum terbagi menjadi tiga macam, \textit{single layer net, multilayer net} dan \textit{competitive layer}. Lebih jauh lagi, metode yang digunakan untuk melatih sebuah model JST dibagi menjadi dua metode, metode yang pertama dikenal dengan \textit{supervised learning} dan yang kedua dikenal dengan  \textit{unsupervised learning} \cite{haykin2009neural}



\section{Rumusan Masalah}
Bagaimana cara memodelkan sebuah model JST dalam menilai kelayakan tugas akhir mahasiswa, dan mengklasifikasi nya apakah termasuk layak atau tidak layak. Selain itu model ini juga harus bisa diintergrasikan atau di implementasikan kedalam sebuah aplikasi. 


\section{Batasan Masalah}
Batasan masalah pada penelitian ini adalah:
\begin{enumerate}
\item Penelitian ini difokuskan pada arsitektur JST \emph{single layer net}
\item Metode yang digunakan hanya \emph{feedforward} dan \emph{backpropagation} dengan fungsi aktivasi \emph{sigmoid}
\item Dataset yang digunakan merujuk kepada jurnal yang akan dilampirkan dan telah dinormalisasi kan\cite{ImplementasiJaringanSyarafTiruanUntukMenilaiKelayakanTugasAkhirMahasiswaStudiKasusDiAmikBukittinggi} 
\item Purwarupa yang dihasilkan akan diimplementasikan kedalam sebuah aplikasi dengan bahasa pemrograman javascript.
\end{enumerate}


\section{Tujuan Penelitian}
Eros reprimique vim no. Alii legendos volutpat in sed, sit enim nemore labores no. No odio decore causae has. Vim te falli libris neglegentur, eam in tempor delectus dignissim, nam hinc dictas an.


\section{Manfaat Penelitian}
Pro omnium incorrupte ea. Elitr eirmod ei qui, ex partem causae disputationi nec. Amet dicant no vis, eum modo omnes quaeque ad, antiopam evertitur reprehendunt pro ut. Nulla inermis est ne. Choro insolens mel ne, eos labitur nusquam eu, nec deserunt reformidans ut. His etiam copiosae principes te, sit brute atqui definiebas id.

Et affert civibus has. Has ne facer accumsan argumentum, apeirian hendrerit persequeris pro ex. Suscipit vivendum sensibus mea at, vim ei hinc numquam, at dicit timeam dissentiet mel. At patrioque intellegebat sea, error argumentum dissentias sea in.



% Baris ini digunakan untuk membantu dalam melakukan sitasi
% Karena diapit dengan comment, maka baris ini akan diabaikan
% oleh compiler LaTeX.
\begin{comment}
\bibliography{daftar-pustaka}
\end{comment}
