%-------------------------------------------------------------------------------
%                      Template Naskah Skripsi
%               	Berdasarkan format JTETI FT UGM
% 						(c) @gunturdputra 2014
%-------------------------------------------------------------------------------

%Template pembuatan naskah skripsi.
\documentclass{jtetiskripsi}

%Untuk prefiks pada daftar gambar dan tabel
\usepackage[titles]{tocloft}
\renewcommand\cftfigpresnum{Gambar\  }
\renewcommand\cfttabpresnum{Tabel\   }

%Untuk hyperlink dan table of content
\usepackage{hyperref}
\newlength{\mylenf}
\settowidth{\mylenf}{\cftfigpresnum}
\setlength{\cftfignumwidth}{\dimexpr\mylenf+2em}
\setlength{\cfttabnumwidth}{\dimexpr\mylenf+2em}

%Untuk Bold Face pada Keterangan Gambar
\usepackage[labelfont=bf]{caption}

%Untuk caption dan subcaption
\usepackage{caption}
\usepackage{subcaption}

%blind text%
\usepackage{blindtext}


\usepackage[mathletters]{ucs}
\usepackage[utf8x]{inputenc}


%-----------------------------------------------------------------
%Disini awal masukan untuk data proposal skripsi
%-----------------------------------------------------------------
\titleind{Laporan Tugas Akhir AI (Jaringan Syaraf Tiruan) Menilai Kelayakan Tugas Akhir Mahasiswa}

\yearsubmit{2019}

\dept{Teknik Informatika}


%-----------------------------------------------------------------
%Disini akhir masukan untuk data proposal skripsi
%-----------------------------------------------------------------

\begin{document}

\cover


%-----------------------------------------------------------------
%Disini awal masukan untuk Prakata
%-----------------------------------------------------------------
\preface
Assalamu'alaikum Wr. Wb.

\vspace{0.5cm}

Puji syukur penyusun panjatkan ke hadirat Allah SWT karena hanya dengan rahmat dan hidayah-Nya, Tugas Akhir ini dapat terselesaikan tanpa halangan berarti. Keberhasilan dalam menyusun laporan Tugas Akhir ini tidak lepas dari bantuan berbagai pihak yang mana dengan tulus dan ikhlas memberikan masukan guna sempurnanya Tugas Akhir ini. Oleh karena itu dalam kesempatan ini, dengan kerendahan hati penyusun mengucapkan terima kasih kepada:

\begin{enumerate}
\item{Ibu Nola Ritha, S.T.,M.Cs., selaku Dosen pengampu mata kulian kecerdasan buatan Universitas Maritim Raja Ali Haji},
\item Teman-teman seperjuangan satu mata kuliah kecerdasan buatan.

\end{enumerate}


Penyusun menyadari bahwa penyusunan Tugas Akhir ini jauh dari sempurna.Besar harapan kami untuk mendapatkan kritik dan saran. Akhir kata penyusun mohon maaf yang sebesar-besarnya apabila ada kekeliruan di dalam penulisan Tugas Akhir ini.

\vspace{0.5cm}

Wassalamu'alaikum Wr. Wb.

\begin{tabular}{p{7.5cm}c}
&Tanjung Pinang, 12 Desember 2019\\
&\\
&\\
&\textbf{Penyusun}
\end{tabular}

\newpage
\noindent Disusun Oleh:
\begin{enumerate}
	\item Rijal Elfikri (180155201009)
	\item Afrio T Putra (1701552010)
	\item Ibnu Isbullah (1701552010)
	\item Raja Azian (170155201010)
\end{enumerate}

%-----------------------------------------------------------------
%Disini akhir masukan untuk muka skripsi
%-----------------------------------------------------------------
\tableofcontents
\addcontentsline{toc}{chapter}{DAFTAR ISI}
\listoffigures
\addcontentsline{toc}{chapter}{DAFTAR GAMBAR}


%-----------------------------------------------------------------
%Disini awal masukan Intisari
%-----------------------------------------------------------------
\begin{abstractind}
Masing-masing mahasiswa telah diberikan buku panduan penulisan tugas akhir untuk
penyusunan tugas akhirnya. Namun masih ditemui beberapa perbedaan pada tugas akhir
mahasiswa yang telah menyelesaikan tugas akhir tersebut. Sehingga, penilaian kelayakan tugas
akhir perlu dilakukan guna memperoleh hasil yang baik dan sesuai dengan format yang ada,
serta layak dipublikasikan sesuai kriteria atau ketentuan yang telah ditetapkan. Untuk
mempercepat proses penilaian dan pengambilan keputusan apakah tugas akhir yang dinilai
tersebut layak atau tidak, tim penilai terkadang hanya melihat hasil secara menyeluruh sebagai
acuan, sehingga hasil penilaianpun tidak bisa dipastikan dengan benar dan tidak objektif.
Penelitian ini akan mengimplementasikan jaringan syaraf tiruan menggunakan algoritma
BackPropagation untuk menilai kelayakan tugas akhir mahasiswa dengan menggunakan
software Matlab 6.1. Pengujian akan dilakukan dengan berbagai pola untuk membandingkan
hasil dari jaringan syaraf tiruan tersebut, agar mendapatkan hasil penilaian yang optimal
apakah tugas akhir yang dinilai tersebut layak atau tidak

\bigskip
\noindent
\textbf{Kata kunci :} backpropagation, Jaringan Syaraf Tiruan, Keputusan, Tugas Akhir.
\end{abstractind}

\begin{abstracteng}
\emph{
Each student has been given a guide book as the guidelines to their final assignment.
But in fact, the students still face the difficulties in following the guidelines and finishing their
final assignment. It caused their final assignment need to be evaluate based on the format. In
increasing the assessment process on how effective and proper of the assignment, usually could
be found through the final conclusion of their final assignment. It may caused some mistaken
assessment in objectivity. This research will be implement the neural network by using
BackPropagation Algorithm in order to know how effective it is, based on the final assignment
of the college students through Matlab 6.1 Software. The assessment will use some methods in
comparing of the neural network, to find the final conclusion about the reasonable of the
research.}

\bigskip
\noindent
\textbf{\emph{Keywords :}} \emph{
	Keywords: Backpropagation, Decision, Final Assignment, Neural Network}.
\end{abstracteng}
%-----------------------------------------------------------------
%Disini akhir masukan Intisari
%-----------------------------------------------------------------

%-----------------------------------------------------------------
%Disini awal masukan untuk Bab
%-----------------------------------------------------------------
%!TEX root = ./ai-jst.tex
%-------------------------------------------------------------------------------
% 								BAB I
% 							LATAR BELAKANG
%-------------------------------------------------------------------------------

\chapter{PENDAHULUAN}

\section{Latar Belakang}
Jaringan syaraf tiruan (JST) atau lebih dikenal dengan \textit{Artificial Neural Network} merupakan sebuah sistem pemrosesan informasi yang memiliki karakteristik kinerja tertentu yang sama dengan jaringan saraf biologis otak manusia \cite{fausett1994fundamentals}. Secara sederhana, jaringan syaraf tiruan adalah sebuah alat pemodelan data statistik non-linier. yang dapat digunakan untuk memodelkan hubungan yang kompleks antara input dan output untuk menemukan pola-pola pada data \cite{wikipedia}. \par
Jaringan syaraf biologis mempunyai tiga komponen utama yaitu: \textit{dendrites, soma} dan \textit{axon}. \textit{dendrites} berkerja sebagai penerima sinyal dari \textit{neuron-neuron}, \textit{soma} berkerja sebagai penjumlah sinyal-sinyal yang diterima, ketika sinyal-sinyal yang diterima dirasa cukup \textit{soma} akan meneruskan sinyal-sinyalnya ke sel-sel yang lain melalui \textit{axon}. Pada bagian \textit{soma} ini, ada kalanya ia meneruskan sinyalnya dan ada kalanya tidak, hal ini bisa dimisalkan dengan proses klasifikasi atau penentuan\cite{fausett1994fundamentals}. Dengan analogi yang demikian model JST dibangun. \par
Penggunaan JST dalam menyelesaikan masalah kian populer, hal ini disebabkan oleh kemampuan JST dalam memodelkan masalah \textit{linear} atau \textit{non-linear} kemudian mempelajari hubungan-hubungan antara \textit{variable-variable} yang diberikan dan pada akhirnya JST mampu memberikan keputusan berdasarkan hasil dari hubungan-hubungan \textit{variable-variable} tadi untuk menyelesaikan permasalahan yang dimaksud. Contoh penerapan JST didalam bidang pendidikan adalah mengklasifikasi tugas akhir mahasiswa apakah masuk dikategori layak atau tidak layak, alih-alih memeriksa satu persatu kelayakan tugas akhir mahasiswa, mengapa tidak melatih sebuah model JST dengan memberikan contoh tugas akhir yang layak dan tugas akhir yang tidak layak, kemudian biarkan JST sendiri mengklarifikasi kelayakan tugas-tugas akhir tersebut. \par 
Arsitektur didalam JST dalam memodelkan jaringan syaraf secara umum terbagi menjadi tiga macam, \textit{single layer net, multilayer net} dan \textit{competitive layer}. Lebih jauh lagi, metode yang digunakan untuk melatih sebuah model JST dibagi menjadi dua metode, metode yang pertama dikenal dengan \textit{supervised learning} dan yang kedua dikenal dengan  \textit{unsupervised learning} \cite{haykin2009neural}



\section{Rumusan Masalah}
Bagaimana cara memodelkan sebuah model JST dalam menilai kelayakan tugas akhir mahasiswa, dan mengklasifikasi nya apakah termasuk layak atau tidak layak. Selain itu model ini juga harus bisa diintergrasikan atau di implementasikan kedalam sebuah aplikasi. 


\section{Batasan Masalah}
Batasan masalah pada penelitian ini adalah:
\begin{enumerate}
\item Penelitian ini difokuskan pada arsitektur JST \emph{single layer net}
\item Metode yang digunakan hanya \emph{feedforward} dan \emph{backpropagation} dengan fungsi aktivasi \emph{sigmoid}
\item Dataset yang digunakan merujuk kepada jurnal yang akan dilampirkan dan telah dinormalisasi kan\cite{ImplementasiJaringanSyarafTiruanUntukMenilaiKelayakanTugasAkhirMahasiswaStudiKasusDiAmikBukittinggi} 
\item Purwarupa yang dihasilkan akan diimplementasikan kedalam sebuah aplikasi dengan bahasa pemrograman javascript.
\end{enumerate}


\section{Tujuan Penelitian}
Tujuan dari penelitian ini adalah mempelajari kemungkinan menyelesaikan masalah klasifikasi dan keputusan terhadap kelayakan tugas akhir mahasiswa dengan model JST.


\section{Manfaat Penelitian}
Dengan mampunya model JST ini memklarifikasi dan memberikan keputusan terhadap tugas akhir mahasiswa maka akan membuka peluang bagi 



% Baris ini digunakan untuk membantu dalam melakukan sitasi
% Karena diapit dengan comment, maka baris ini akan diabaikan
% oleh compiler LaTeX.
\begin{comment}
\bibliography{daftar-pustaka}
\end{comment}


%!TEX root = ./ai-jst.tex
%-------------------------------------------------------------------------------
%                            BAB II
%               TINJAUAN PUSTAKA DAN DASAR TEORI
%-------------------------------------------------------------------------------

\chapter{KAJIAN LITERATUR}                

\section{Tinjauan Pustaka}
  Lorem ipsum is a pseudo-Latin text used in web design, typography, layout, and printing in place of English to emphasise design elements over content. It's also called placeholder (or filler) text. It's a convenient tool for mock-ups. It helps to outline the visual elements of a document or presentation, eg typography, font, or layout. Lorem ipsum is mostly a part of a Latin text by the classical author and philospher Cicero. Its words and letters have been changed by addition or removal, so to deliberately render its content nonsensical; it's not genuine, correct, or comprehensible Latin anymore. While lorem ipsum's still resembles classical Latin, it actually has no meaning whatsoever. As Cicero's text doesn't contain the letters K, W, or Z, alien to latin, these, and others are often inserted randomly to mimic the typographic appearence of European languages, as are digraphs not to be found in the original.

\section{Landasan Teori}
 \blindtext[2]
% Baris ini digunakan untuk membantu dalam melakukan sitasi
% Karena diapit dengan comment, maka baris ini akan diabaikan
% oleh compiler LaTeX.
\begin{comment}
\bibliography{daftar-pustaka}
\end{comment}


%!TEX root = ./ai-jst.tex
%-------------------------------------------------------------------------------
%                            BAB III
%                     METODOLOGI PENELITIAN
%-------------------------------------------------------------------------------

\chapter{METODOLOGI PENELITIAN}                

\section{Waktu dan Tempat Penelitian}
  Penelitian ini dilakukan dengan mengikuti jurnal Implementasi jaringan syaraf tiruan untuk menilai kelayakan tugas akhir mahasiswa (studi kasus di amik bukittinggi)
yang dioublikasi oleh Jurnal Teknologi Informasi dan Komunikasi Digital Zone, Volume 8, Nomor 1, Tahun 2017.

\section{Analisa dan Perancangan}
 Adapun Analisa dan Perancangan didalam penelitian ini adalah:
\begin{enumerate}
	\item  Mengumpulkan data penilain terhadap Tugas Akhir Mahasiswa.
	\item Mendefinisikan \emph{input} dan \emph{target}.
	\item Membagi data menjadi dua kelompok yaitu, data latih dan data uji.
	\item Memulai proses pembelajaran prediksi dengan metode \emph{Artificial Neural Network} dengan langkah sebagai berikut:
	\begin{enumerate}[a.]
		\item \emph{ANN} dilatih dengan cara menginisialisasi bobot \emph{weight} awal.
		\item Menentukan fungsi aktivasi , karena ini hanya menampilkan dua keputusan maka fungsi aktivasi yang digunakan pada layar tersembunyi dan layar output adalah fungsi aktivasi sigmoid. Yang dirumuskan sebagai berikut:
		\begin{enumerate}[i.]
			\item Setiap unit tersembunyi ($ z_i = z1,...,p $) menjumlahkan bobot sinyal \emph{input} $$ Z_{in_{j}} = v_{j0} + \Sigma^{n}_{i=1} x_{i} v_{ji} $$.
			\item Dengan menerapkan fungsi aktivasi sigmoid didapatkan keluaran pada layar tersembunyi z $$ Z_{j} = f (Z_{in_{j}}) = \frac{1}{1+e^{-z_{in_{j}}}} $$
			\item Hitung semua keluaran jaringan di unit $y_{k}$ (k= 1,2,... m) $$ y_{in_{k}} = w_{k0} + \Sigma^{p}_{j=1} Z_{i} W_{kj} $$
			\item Dengan menerapkan fungsi aktivasi sigmoid biner didapatkan keluaran pada output y $$ y_{k} = f(y_{in_{j}}) = \frac{1}{1+e^{-y_{in_{k}}}} $$
		\end{enumerate}
	
		\item Menggunakan metode backpropagation sebagai pelatihan arah maju atau sebagai pelatihan arah mundur sampai tingkat kesalahan dan jumlah iterasi sebesar 1000 dan target error sebesar 0.001.
		\item  Mendapatkan Arsitektur Jaringannya.
		\item  Pengujian data Setelah data sudah disiapkan, langkah selanjutnya adalah mulai melakukan pengujian terhadap data-data tugas akhir mahasiswa.
	\end{enumerate}
\end{enumerate}
% Baris ini digunakan untuk membantu dalam melakukan sitasi
% Karena diapit dengan comment, maka baris ini akan diabaikan
% oleh compiler LaTeX.
\begin{comment}
\bibliography{daftar-pustaka}
\end{comment}


%!TEX root = ./ai-jst.tex
%-------------------------------------------------------------------------------
%                            BAB IV
%               	ANALISAN dan PEMBAHASAN
%-------------------------------------------------------------------------------

\chapter{ANALISA DAN PEMBAHASAN}                

\section{Bahasan 1}
  Lorem ipsum is a pseudo-Latin text used in web design, typography, layout, and printing in place of English to emphasise design elements over content. It's also called placeholder (or filler) text. It's a convenient tool for mock-ups. It helps to outline the visual elements of a document or presentation, eg typography, font, or layout. Lorem ipsum is mostly a part of a Latin text by the classical author and philospher Cicero. Its words and letters have been changed by addition or removal, so to deliberately render its content nonsensical; it's not genuine, correct, or comprehensible Latin anymore. While lorem ipsum's still resembles classical Latin, it actually has no meaning whatsoever. As Cicero's text doesn't contain the letters K, W, or Z, alien to latin, these, and others are often inserted randomly to mimic the typographic appearence of European languages, as are digraphs not to be found in the original.

\section{Bahasan 2}
 \blindtext[2]
% Baris ini digunakan untuk membantu dalam melakukan sitasi
% Karena diapit dengan comment, maka baris ini akan diabaikan
% oleh compiler LaTeX.
\begin{comment}
\bibliography{daftar-pustaka}
\end{comment}


%!TEX root = ./ai-jst.tex
%-------------------------------------------------------------------------------
%                            BAB V
%               	 KESIMPULAN dan SARAN
%-------------------------------------------------------------------------------

\chapter{KESIMPULAN DAN SARAN}                

\section{Kesimpulan}
  Lorem ipsum is a pseudo-Latin text used in web design, typography, layout, and printing in place of English to emphasise design elements over content. It's also called placeholder (or filler) text. It's a convenient tool for mock-ups. It helps to outline the visual elements of a document or presentation, eg typography, font, or layout. Lorem ipsum is mostly a part of a Latin text by the classical author and philospher Cicero. Its words and letters have been changed by addition or removal, so to deliberately render its content nonsensical; it's not genuine, correct, or comprehensible Latin anymore. While lorem ipsum's still resembles classical Latin, it actually has no meaning whatsoever. As Cicero's text doesn't contain the letters K, W, or Z, alien to latin, these, and others are often inserted randomly to mimic the typographic appearence of European languages, as are digraphs not to be found in the original.

\section{Saran}
 \blindtext[2]
% Baris ini digunakan untuk membantu dalam melakukan sitasi
% Karena diapit dengan comment, maka baris ini akan diabaikan
% oleh compiler LaTeX.
\begin{comment}
\bibliography{daftar-pustaka}
\end{comment}


%-----------------------------------------------------------------
%Disini akhir masukan Bab
%-----------------------------------------------------------------


%-----------------------------------------------------------------
% Disini awal masukan untuk Daftar Pustaka
% - Daftar pustaka diambil dari file .bib yang ada pada folder ini
%   juga.
% - Untuk memudahkan dalam memanajemen dan menggenerate file .bib
%   gunakan reference manager seperti Mendeley, Zotero, EndNote,
%   dll.
%-----------------------------------------------------------------
\bibliography{IEEEabrv,daftar-pustaka}
\addcontentsline{toc}{chapter}{DAFTAR PUSTAKA}
%-----------------------------------------------------------------
%Disini akhir masukan Daftar Pustaka
%-----------------------------------------------------------------

\end{document}