%-------------------------------------------------------------------------------
%                      Template Naskah Skripsi
%               	Berdasarkan format JTETI FT UGM
% 						(c) @gunturdputra 2014
%-------------------------------------------------------------------------------

%Template pembuatan naskah skripsi.
\documentclass{jtetiskripsi}

%Untuk prefiks pada daftar gambar dan tabel
\usepackage[titles]{tocloft}
\renewcommand\cftfigpresnum{Gambar\  }
\renewcommand\cfttabpresnum{Tabel\   }

%Untuk hyperlink dan table of content
\usepackage{hyperref}
\newlength{\mylenf}
\settowidth{\mylenf}{\cftfigpresnum}
\setlength{\cftfignumwidth}{\dimexpr\mylenf+2em}
\setlength{\cfttabnumwidth}{\dimexpr\mylenf+2em}

%Untuk Bold Face pada Keterangan Gambar
\usepackage[labelfont=bf]{caption}

%Untuk caption dan subcaption
\usepackage{caption}
\usepackage{subcaption}

%blind text%
\usepackage{blindtext}


\usepackage[mathletters]{ucs}
\usepackage[utf8x]{inputenc}


%-----------------------------------------------------------------
%Disini awal masukan untuk data proposal skripsi
%-----------------------------------------------------------------
\titleind{TEMPLATE SKRIPSI JTETI DENGAN MENGGUNAKAN \emph{TYPESETTING} \LaTeX}

\fullname{GUNTUR D PUTRA}

\idnum{09/123456/TK/12345}

\approvaldate{3 Februari 2014}

\degree{Sarjana Teknik Elektro}

\yearsubmit{2014}

\program{Teknik Elektro}

\dept{Teknik Elektro dan Teknologi Informasi}

\firstsupervisor{Sigit Basuki Wibowo, S.T., M.Eng.}
\firstnip{1976 0501 2002 12 1 002}

\secondsupervisor{Bimo Sunarfri Hantono, S.T., M.Eng.}
\secondnip{1977 0131 2002 12 1 003}

%-----------------------------------------------------------------
%Disini akhir masukan untuk data proposal skripsi
%-----------------------------------------------------------------

\begin{document}

\cover


%-----------------------------------------------------------------
%Disini awal masukan untuk Prakata
%-----------------------------------------------------------------
\preface
Assalamu'alaikum Wr. Wb.

\vspace{0.5cm}

Puji syukur penulis panjatkan ke hadirat Allah SWT karena hanya dengan rahmat dan hidayah-Nya, Tugas Akhir ini dapat terselesaikan tanpa halangan berarti. Keberhasilan dalam menyusun laporan Tugas Akhir ini tidak lepas dari bantuan berbagai pihak yang mana dengan tulus dan ikhlas memberikan masukan guna sempurnanya Tugas Akhir ini. Oleh karena itu dalam kesempatan ini, dengan kerendahan hati penulis mengucapkan terima kasih kepada:

\begin{enumerate}
\item{Bapak Sarjiya, S.T., M.T., Ph.D., selaku Ketua Jurusan Teknik Elektro dan Teknologi Informasi Fakultas Teknik Universitas Gadjah Mada,}
\item{Bapak Sigit Basuki Wibowo, S.T., M.Eng. selaku dosen pembimbing pertama yang telah memberikan banyak bantuan, bimbingan, serta arahan dalam Tugas Akhir ini,}
\item{Bapak Bimo Sunarfri Hantono, S.T., M.Eng. selaku dosen pembimbing kedua yang juga telah memberikan banyak bantuan, bimbingan, serta arahan dalam Tugas Akhir dan kegiatan-kegiatan yang lain,}
\item{Bapak Warsun Najib, S.T., M.Sc. selaku dosen pembimbing akademis penulis dan juga dosen pembimbing lapangan penulis pada KKN-PPM UGM 2013 Unit SLM07,}
\item{Seluruh Dosen di Jurusan Teknik Elektro dan Teknologi Informasi FT UGM, yang tidak bisa disebutkan satu-satu, atas ilmu dan bimbingannya selama penulis berkuliah di JTETI,}
\item{Ibu dan Bapak yang selama ini telah sabar membimbing, mengarahkan, dan mendoakan penulis tanpa kenal lelah untuk selama-lamanya, dan}
\item{Cantumkan pihak-pihak lain yang ingin anda berikan ucapan terimakasih.}
\end{enumerate}


\newpage
Penulis menyadari bahwa penyusunan Tugas Akhir ini jauh dari sempurna. Kritik dan saran dapat ditujukan langsung pada e-mail atau \emph{mention} langsung pada akun \emph{twitter} saya. Akhir kata penulis mohon maaf yang sebesar-besarnya apabila ada kekeliruan di dalam penulisan Tugas Akhir ini.

\vspace{0.5cm}

Wassalamu'alaikum Wr. Wb.

\begin{tabular}{p{7.5cm}c}
&Yogyakarta, 15 Januari 2014\\
&\\
&\\
&\textbf{Penulis}
\end{tabular}

%-----------------------------------------------------------------
%Disini akhir masukan untuk muka skripsi
%-----------------------------------------------------------------
\tableofcontents
\addcontentsline{toc}{chapter}{DAFTAR ISI}
\listoftables
\addcontentsline{toc}{chapter}{DAFTAR TABEL}
\listoffigures
\addcontentsline{toc}{chapter}{DAFTAR GAMBAR}


%-----------------------------------------------------------------
%Disini awal masukan Intisari
%-----------------------------------------------------------------
\begin{abstractind}
Lorem ipsum dolor sit amet, consectetur adipisicing elit, sed do eiusmod tempor incididunt ut labore et dolore magna aliqua. Ut enim ad minim veniam, quis nostrud exercitation ullamco laboris nisi ut aliquip ex ea commodo consequat. Duis aute irure dolor in reprehenderit in voluptate velit esse cillum dolore eu fugiat nulla pariatur. Excepteur sint occaecat cupidatat non proident, sunt in culpa qui officia deserunt mollit anim id est laborum.

Sed ut perspiciatis unde omnis iste natus error sit voluptatem accusantium doloremque laudantium, totam rem aperiam, eaque ipsa quae ab illo inventore veritatis et quasi architecto beatae vitae dicta sunt explicabo. Nemo enim ipsam voluptatem quia voluptas sit aspernatur aut odit aut fugit, sed quia consequuntur magni dolores eos qui ratione voluptatem sequi nesciunt.


\bigskip
\noindent
\textbf{Kata kunci :} \emph{wireless sensor network}, \emph{Internet Protocol}, WiFi, interoperabilitas.
\end{abstractind}

\begin{abstracteng}
\emph{
Lorem ipsum dolor sit amet, consectetur adipisicing elit, sed do eiusmod tempor incididunt ut labore et dolore magna aliqua. Ut enim ad minim veniam, quis nostrud exercitation ullamco laboris nisi ut aliquip ex ea commodo consequat. Duis aute irure dolor in reprehenderit in voluptate velit esse cillum dolore eu fugiat nulla pariatur. Excepteur sint occaecat cupidatat non proident, sunt in culpa qui officia deserunt mollit anim id est laborum.}

\emph{Sed ut perspiciatis unde omnis iste natus error sit voluptatem accusantium doloremque laudantium, totam rem aperiam, eaque ipsa quae ab illo inventore veritatis et quasi architecto beatae vitae dicta sunt explicabo. Nemo enim ipsam voluptatem quia voluptas sit aspernatur aut odit aut fugit, sed quia consequuntur magni dolores eos qui ratione voluptatem sequi nesciunt.}

\bigskip
\noindent
\textbf{\emph{Keywords :}} \emph{wireless sensor network, Internet Protokol, WiFi, interoperability}.
\end{abstracteng}
%-----------------------------------------------------------------
%Disini akhir masukan Intisari
%-----------------------------------------------------------------

%-----------------------------------------------------------------
%Disini awal masukan untuk Bab
%-----------------------------------------------------------------
%!TEX root = ./ai-jst.tex
%-------------------------------------------------------------------------------
% 								BAB I
% 							LATAR BELAKANG
%-------------------------------------------------------------------------------

\chapter{PENDAHULUAN}

\section{Latar Belakang}
Jaringan syaraf tiruan (JST) atau lebih dikenal dengan \textit{Artificial Neural Network} merupakan sebuah sistem pemrosesan informasi yang memiliki karakteristik kinerja tertentu yang sama dengan jaringan saraf biologis otak manusia \cite{fausett1994fundamentals}. Secara sederhana, jaringan syaraf tiruan adalah sebuah alat pemodelan data statistik non-linier. yang dapat digunakan untuk memodelkan hubungan yang kompleks antara input dan output untuk menemukan pola-pola pada data \cite{wikipedia}. \par
Jaringan syaraf biologis mempunyai tiga komponen utama yaitu: \textit{dendrites, soma} dan \textit{axon}. \textit{dendrites} berkerja sebagai penerima sinyal dari \textit{neuron-neuron}, \textit{soma} berkerja sebagai penjumlah sinyal-sinyal yang diterima, ketika sinyal-sinyal yang diterima dirasa cukup \textit{soma} akan meneruskan sinyal-sinyalnya ke sel-sel yang lain melalui \textit{axon}. Pada bagian \textit{soma} ini, ada kalanya ia meneruskan sinyalnya dan ada kalanya tidak, hal ini bisa dimisalkan dengan proses klasifikasi atau penentuan\cite{fausett1994fundamentals}. Dengan analogi yang demikian model JST dibangun. \par
Penggunaan JST dalam menyelesaikan masalah kian populer, hal ini disebabkan oleh kemampuan JST dalam memodelkan masalah \textit{linear} atau \textit{non-linear} kemudian mempelajari hubungan-hubungan antara \textit{variable-variable} yang diberikan dan pada akhirnya JST mampu memberikan keputusan berdasarkan hasil dari hubungan-hubungan \textit{variable-variable} tadi untuk menyelesaikan permasalahan yang dimaksud. Contoh penerapan JST didalam bidang pendidikan adalah mengklasifikasi tugas akhir mahasiswa apakah masuk dikategori layak atau tidak layak, alih-alih memeriksa satu persatu kelayakan tugas akhir mahasiswa, mengapa tidak melatih sebuah model JST dengan memberikan contoh tugas akhir yang layak dan tugas akhir yang tidak layak, kemudian biarkan JST sendiri mengklarifikasi kelayakan tugas-tugas akhir tersebut. \par 
Arsitektur didalam JST dalam memodelkan jaringan syaraf secara umum terbagi menjadi tiga macam, \textit{single layer net, multilayer net} dan \textit{competitive layer}. Lebih jauh lagi, metode yang digunakan untuk melatih sebuah model JST dibagi menjadi dua metode, metode yang pertama dikenal dengan \textit{supervised learning} dan yang kedua dikenal dengan  \textit{unsupervised learning} \cite{haykin2009neural}



\section{Rumusan Masalah}
Bagaimana cara memodelkan sebuah model JST dalam menilai kelayakan tugas akhir mahasiswa, dan mengklasifikasi nya apakah termasuk layak atau tidak layak. Selain itu model ini juga harus bisa diintergrasikan atau di implementasikan kedalam sebuah aplikasi. 


\section{Batasan Masalah}
Batasan masalah pada penelitian ini adalah:
\begin{enumerate}
\item Penelitian ini difokuskan pada arsitektur JST \emph{single layer net}
\item Metode yang digunakan hanya \emph{feedforward} dan \emph{backpropagation} dengan fungsi aktivasi \emph{sigmoid}
\item Dataset yang digunakan merujuk kepada jurnal yang akan dilampirkan dan telah dinormalisasi kan\cite{ImplementasiJaringanSyarafTiruanUntukMenilaiKelayakanTugasAkhirMahasiswaStudiKasusDiAmikBukittinggi} 
\item Purwarupa yang dihasilkan akan diimplementasikan kedalam sebuah aplikasi dengan bahasa pemrograman javascript.
\end{enumerate}


\section{Tujuan Penelitian}
Tujuan dari penelitian ini adalah mempelajari kemungkinan menyelesaikan masalah klasifikasi dan keputusan terhadap kelayakan tugas akhir mahasiswa dengan model JST.


\section{Manfaat Penelitian}
Dengan mampunya model JST ini memklarifikasi dan memberikan keputusan terhadap tugas akhir mahasiswa maka akan membuka peluang bagi 



% Baris ini digunakan untuk membantu dalam melakukan sitasi
% Karena diapit dengan comment, maka baris ini akan diabaikan
% oleh compiler LaTeX.
\begin{comment}
\bibliography{daftar-pustaka}
\end{comment}


%!TEX root = ./ai-jst.tex
%-------------------------------------------------------------------------------
%                            BAB II
%               TINJAUAN PUSTAKA DAN DASAR TEORI
%-------------------------------------------------------------------------------

\chapter{KAJIAN LITERATUR}                

\section{Tinjauan Pustaka}
  Lorem ipsum is a pseudo-Latin text used in web design, typography, layout, and printing in place of English to emphasise design elements over content. It's also called placeholder (or filler) text. It's a convenient tool for mock-ups. It helps to outline the visual elements of a document or presentation, eg typography, font, or layout. Lorem ipsum is mostly a part of a Latin text by the classical author and philospher Cicero. Its words and letters have been changed by addition or removal, so to deliberately render its content nonsensical; it's not genuine, correct, or comprehensible Latin anymore. While lorem ipsum's still resembles classical Latin, it actually has no meaning whatsoever. As Cicero's text doesn't contain the letters K, W, or Z, alien to latin, these, and others are often inserted randomly to mimic the typographic appearence of European languages, as are digraphs not to be found in the original.

\section{Landasan Teori}
 \blindtext[2]
% Baris ini digunakan untuk membantu dalam melakukan sitasi
% Karena diapit dengan comment, maka baris ini akan diabaikan
% oleh compiler LaTeX.
\begin{comment}
\bibliography{daftar-pustaka}
\end{comment}


%!TEX root = ./ai-jst.tex
%-------------------------------------------------------------------------------
%                            BAB III
%                     METODOLOGI PENELITIAN
%-------------------------------------------------------------------------------

\chapter{METODOLOGI PENELITIAN}                

\section{Waktu dan Tempat Penelitian}
  Penelitian ini dilakukan dengan mengikuti jurnal Implementasi jaringan syaraf tiruan untuk menilai kelayakan tugas akhir mahasiswa (studi kasus di amik bukittinggi)
yang dioublikasi oleh Jurnal Teknologi Informasi dan Komunikasi Digital Zone, Volume 8, Nomor 1, Tahun 2017.

\section{Analisa dan Perancangan}
 Adapun Analisa dan Perancangan didalam penelitian ini adalah:
\begin{enumerate}
	\item  Mengumpulkan data penilain terhadap Tugas Akhir Mahasiswa.
	\item Mendefinisikan \emph{input} dan \emph{target}.
	\item Membagi data menjadi dua kelompok yaitu, data latih dan data uji.
	\item Memulai proses pembelajaran prediksi dengan metode \emph{Artificial Neural Network} dengan langkah sebagai berikut:
	\begin{enumerate}[a.]
		\item \emph{ANN} dilatih dengan cara menginisialisasi bobot \emph{weight} awal.
		\item Menentukan fungsi aktivasi , karena ini hanya menampilkan dua keputusan maka fungsi aktivasi yang digunakan pada layar tersembunyi dan layar output adalah fungsi aktivasi sigmoid. Yang dirumuskan sebagai berikut:
		\begin{enumerate}[i.]
			\item Setiap unit tersembunyi ($ z_i = z1,...,p $) menjumlahkan bobot sinyal \emph{input} $$ Z_{in_{j}} = v_{j0} + \Sigma^{n}_{i=1} x_{i} v_{ji} $$.
			\item Dengan menerapkan fungsi aktivasi sigmoid didapatkan keluaran pada layar tersembunyi z $$ Z_{j} = f (Z_{in_{j}}) = \frac{1}{1+e^{-z_{in_{j}}}} $$
			\item Hitung semua keluaran jaringan di unit $y_{k}$ (k= 1,2,... m) $$ y_{in_{k}} = w_{k0} + \Sigma^{p}_{j=1} Z_{i} W_{kj} $$
			\item Dengan menerapkan fungsi aktivasi sigmoid biner didapatkan keluaran pada output y $$ y_{k} = f(y_{in_{j}}) = \frac{1}{1+e^{-y_{in_{k}}}} $$
		\end{enumerate}
	
		\item Menggunakan metode backpropagation sebagai pelatihan arah maju atau sebagai pelatihan arah mundur sampai tingkat kesalahan dan jumlah iterasi sebesar 1000 dan target error sebesar 0.001.
		\item  Mendapatkan Arsitektur Jaringannya.
		\item  Pengujian data Setelah data sudah disiapkan, langkah selanjutnya adalah mulai melakukan pengujian terhadap data-data tugas akhir mahasiswa.
	\end{enumerate}
\end{enumerate}
% Baris ini digunakan untuk membantu dalam melakukan sitasi
% Karena diapit dengan comment, maka baris ini akan diabaikan
% oleh compiler LaTeX.
\begin{comment}
\bibliography{daftar-pustaka}
\end{comment}


%!TEX root = ./ai-jst.tex
%-------------------------------------------------------------------------------
%                            BAB IV
%               	ANALISAN dan PEMBAHASAN
%-------------------------------------------------------------------------------

\chapter{ANALISA DAN PEMBAHASAN}                

\section{Analisa Kebutuhan Sistem}
 Bagian ini menjelaskan hal-hal yang terkait dengan pengembangan aplikasi sebelum penulisan \emph{source code}.
 \subsection{\emph{Flow Chart}}
 Berikut merupakan gambaran umum bagaimana aplikasi ini berkerja. 
 
 	\begin{figure}[H]
	 	\centering
	 	\includegraphics[height=13cm]{gambar/flowchart-jst}
	 	\caption{\emph{Flowchart dari sistem}}
	 	\label{JST-3}
	 \end{figure}
 
 Dimana \emph{input}an x1,x2,x3,x4,x5,x6, dan x7 merupakan variable dari bagian-bagian Tugas Akhir Mahasiswa yang akan lebih dijelaskan pada pembahasan selajutnya.
 
 \subsection{Kebutuhan Kode Sumber \emph{Source Code}}
 Pada pengembangan model jaringan syaraf tiruan ini kami mengunakan sebuah \emph{framework} bahasa pemrograman javascript yaitu Brain.js untuk pemodelan arsitektur JSTnya, sedangkan untuk tampilan kami mengunakan bahasa \emph{markup} HTML dan sebagai \emph{Cascading Stylesheet}nya kami mengunakan pustaka dari materialize css.

\section{Hasil Penelitian}
Data yang kami gunakan didalam penelitian ini merupakan data yang kami dapat dari jurnal yang menjadi titik acuan dan referensi kami \cite{ZeksonArizonaMatondang}.
\subsection{Profil Data}
Data yang dicantumkan pada jurnal yang menjadi referensi kami merupakan data penilaian terhadap Tugas Akhir Mahasiswa yang telah dinormalisasikan dan dibagi menjadi beberapa kriteria sebagaimana tabel berikut.

\begin{center}
	\begin{tabular}{|c|c|c|c|}
		\hline
		No & Variabel & Kriteria \\
		\hline
		1.& x1 & Abstrak \\
		\hline
		2.& x2 & Pendahuluan \\
		\hline
		3.& x3 & Landasan Teori \\
		\hline
		4.& x4 & Metode Penelitian \\
		\hline
		5.& x5 & Hasil dan Pembahasan \\
		\hline
		6.& x6 & Penutup \\
		\hline
		7.& x7 & Daftar Pustaka \\
		\hline
	\end{tabular}
\end{center}
\par 
Data-data yang telah dinormalisasi sebagaimana yang tercantum didalam jurnal terkait bisa dilihat pada tabel berikut.

\begin{center}
	\begin{tabular}{|c|c|c|c|c|c|c|c|c|}
		\hline
		p & x1 & x2 & x3 & x4 & x5 & x6 & x7 & t \\
		\hline
		\hline
		p1 & 0,3 & 0,7 & 0,3 & 0,7 & 0,7 & 0,7 & 0,3 & 1 \\
		\hline
		p2  & 0,7 & 0,7 & 0,7 & 0,7 & 0,9 & 0,7 & 0,7 & 1 \\
		\hline
		p3  & 0,3 & 0,7 & 0,3 & 0,7 & 0,7 & 0,7 & 0,3 & 1 \\
		\hline
		p4  & 0,7 & 0,7 & 0,7 & 0,7 & 0,7 & 0,7 & 0,7 & 1 \\
		\hline
		p5  & 0,3 & 0,7 & 0,7 & 0,9 & 0,7 & 0,7 & 0,7 & 1 \\
		\hline
		p6  & 0,3 & 0,7 & 0,7 & 0,7 & 0,7 & 0,7 & 0,7 & 1 \\
		\hline
		p7  & 0,3 & 0,3 & 0,3 & 0,7 & 0,7 & 0,7 & 0,3 & 0 \\
		\hline
		p8  & 0,7 & 0,3 & 0,7 & 0,7 & 0,7 & 0,7 & 0,3 & 1 \\
		\hline
		p9  & 0,3 & 0,3 & 0,3 & 0,7 & 0,3 & 0,3 & 0,3 & 0 \\
		\hline
		p10 &  0,1 & 0,3 & 0,3 & 0,7 & 0,3 & 0,3 & 0,3 & 0 \\
		\hline
		p11 &  0,3 & 0,3 & 0,7 & 0,7 & 0,7 & 0,7 & 0,7 & 1 \\
		\hline
		p12 &  0,7 & 0,9 & 0,7 & 0,9 & 0,9 & 0,7 & 0,7 & 1 \\
		\hline
		p13 &  0,7 & 0,9 & 0,7 & 0,9 & 0,9 & 0,7 & 0,7 & 1 \\
		\hline
		p14 &  0,3 & 0,3 & 0,7 & 0,7 & 0,7 & 0,7 & 0,3 & 1 \\
		\hline
		p15 &  0,7 & 0,9 & 0,7 & 0,7 & 0,7 & 0,9 & 0,7 & 1 \\
		\hline
		p16 &  0,3 & 0,7 & 0,9 & 0,7 & 0,9 & 0,7 & 0,3 & 1 \\
		\hline
	\end{tabular}
\par
\bigskip
Data latih yang telah dinormalisasikan
\end{center}



\begin{center}
	\begin{tabular}{|c|c|c|c|c|c|c|c|c|}
		\hline
		p36 & 0,7 & 0,7 & 0,7 & 0,7 & 0,9 & 0,7 & 0,7 & 1 \\
		\hline
		p37 & 0,3 & 0,7 & 0,3 & 0,7 & 0,7 & 0,7 & 0,7 & 1 \\
		\hline
		p38 & 0,3 & 0,3 & 0,3 & 0,7 & 0,7 & 0,3 & 0,3 & 0 \\
		\hline
		p39 & 0,7 & 0,7 & 0,7 & 0,9 & 0,7 & 0,7 & 0,7 & 1 \\
		\hline
		p40 & 0,3 & 0,7 & 0,3 & 0,7 & 0,7 & 0,7 & 0,7 & 1 \\
		\hline
		p41 & 0,7 & 0,3 & 0,7 & 0,7 & 0,7 & 0,7 & 0,3 & 1 \\
		\hline
		p42 & 0,7 & 0,7 & 0,7 & 0,7 & 0,9 & 0,7 & 0,7 & 1 \\
		\hline
		p43 & 0,3 & 0,7 & 0,7 & 0,7 & 0,7 & 0,7 & 0,3 & 1 \\
		\hline
		p44 & 0,3 & 0,3 & 0,3 & 0,7 & 0,7 & 0,7 & 0,3 & 0 \\
		\hline
		p45 & 0,3 & 0,7 & 0,7 & 0,7 & 0,7 & 0,7 & 0,7 & 1 \\
		\hline
		p46 & 0,3 & 0,7 & 0,3 & 0,7 & 0,3 & 0,3 & 0,3 & 0 \\
		\hline
		p47 & 0,3 & 0,7 & 0,3 & 0,7 & 0,3 & 0,7 & 0,3 & 0 \\
		\hline
		p48 & 0,7 & 0,7 & 0,7 & 0,7 & 0,9 & 0,7 & 0,7 & 1 \\
		\hline
		p49 & 0,3 & 0,7 & 0,7 & 0,7 & 0,3 & 0,7 & 0,7 & 1 \\
		\hline
		p50 & 0,3 & 0,7 & 0,7 & 0,7 & 0,7 & 0,7 & 0,3 & 1 \\
		\hline
	\end{tabular}
	\par
	\bigskip
	Data uji yang telah dinormalisasikan
\end{center}


Dimana p merupakan simbol dari tugas akhir mahasiswa itu sendiri dan t sebagai target, dengan 1 merupakan untuk kategori layak dan 0 untuk kategori tidak layak.

\subsection{Pembahsan dan Perancangan Arsitektur \emph{Neural Network}}
Metode Jaringan Syaraf Tiruan yang digunakan untuk menilai kelayakan tugas akhir mahasiswa adalah Jaringan Syaraf Tiruan Backpropagation. Metode jaringan syaraf tiruan ini memiliki beberapa lapisan, yaitu lapisan masukan, lapisan keluaran dan beberapa lapisan tersembunyi. Untuk mempermudah pengerjaannya kami mengunakan \emph{framework} jabascript yang memang ditujukan untui pengerjaan arsitektur jaringan syaraf tiruan dengan metode backpropagation yaitu brain.js\footnote{https://brain.js.org}
\par
Arsitektur jaringan syaraf tiruan yang akan kami uji sebagai berikut.
\begin{enumerate}
	\item 7-1-1 (dengan 1 \emph{neuron}).
	\item 7-1-1 (dengan 5 \emph{neuron}).
	\item 7-5-3-1.
	\item 7-3-1.
\end{enumerate} 
Dari hasil \emph{training} dilakukan kepada model JST ini dengan data latih maka didapati hasil sebagai berikut.

\begin{center}
	\begin{tabular}{|c|c|c|c|c|c|c|c|c|}
		\hline
		No & Nilai Error & Jumlah \emph{Hidden Layer} & Jumlah neuron \emph{hidden layer} & \emph{Iteration}  \\
		\hline
		\hline
		1 & 0.00491 & 1 & 1 & 400  \\
		\hline
		2 & 0.00499 & 1 & 5 & 401  \\
		\hline
		3 & 0.00498 & 2 & 8 (5+3) & 433  \\
		\hline
		4 & 0.00497 & 1 & 3 & 388  \\
		\hline
	\end{tabular}
	\par
	\bigskip
	Nilai \emph{error} dan banyaknya \emph{iterasi} untuk masing-masing arsitektur jaringan
\end{center}

\par 
Kami memilih untuk menggunakan arsitektur jaringan ke-4 dimana didapati hasil dari data uji sebagai berikut.

\begin{center}
	\begin{tabular}{|c|c|c|c|c|c|c|c|c|}
		\hline
		No & Target & Hasil Prediksi  \\
		\hline
		\hline
		1 & 1 & 0.995  \\
		\hline
		2 & 1 & 0.960 \\
		\hline
		3 & 0 & 0.260 \\
		\hline
		4 & 1 & 0.891 \\
		\hline
		5 & 1 & 0.976 \\
		\hline
		6 & 1 & 0.953 \\
		\hline
		7 & 1 & 0.966 \\
		\hline
		8 & 0 & 0.341 \\
		\hline
		9 & 1 & 0.917 \\
		\hline
		10 & 0 & 0.516 \\
		\hline
		11 & 0 & 0.257 \\
		\hline
		12 & 1 & 0.910 \\
		\hline
		13 & 1 & 0.876 \\
		\hline
		14 & 1 & 0.967 \\
		\hline
	\end{tabular}
	\par
	\bigskip
	Target dan hasil prediksi untuk masing-masing data uji
\end{center}

\par 
Konsep perhitungan model jaringan syaraf tiruan secara manual yang kami kerjakan dapat disimak pada berikut.
 \begin{enumerate}
 	\item Parameter yang kami gunakan dapat dihitung melalui $ input x jumlah-  \emph{hidden layer} + \emph{Hidden layer} $ dimana \emph{ Hidden layer} dan \emph{output layer} memiliki tambahan \emph{“input”} yang biasa disebut dengan bias. Sehingga didapati $ (7 x 3 + 3) + (3 x 1 + 1) $, sehingga total parameter yang digunakan adalah 28 parameter.
 	
 	\item Memulai  \emph{feedforward training}, kita akan memluai \emph{training} dengan menjumlah setiap \emph{input} dan \emph{weight} di tiap-tiap parameter dimana hal ini bisa dikerjakan dengan $$ j_{n_{in}} = [input x w_{n}] + b_{1} $$
 	
 	\item Nilai yang kita dapati setelah melalui proses diatas akan kita masukkan kedalam \emph{activation function} diaman kita mengunakan \emph{sigmoid} sehingga:
 	$$ Sigmoid(k_{n_{in}}) = \frac{1}{1 + e^{k_{n_{in}}}} $$
 	\item ketika data sudah mengalir ke \emph{neuron} paling akhir \emph{output} kita akan menghitung nilai \emph{error}nya dengan $$ Loss = \frac{1}{2}(output-O_{out})^{2} $$
 	
 	\item Pada tahap ini kita akan memulai aktivitas backpropagation untuk mengubah nilai tiap-tiap \emph{weight} dan bias sehingga kita mendapatkan nilai error yang paling terkecil.
 \end{enumerate}


% Baris ini digunakan untuk membantu dalam melakukan sitasi
% Karena diapit dengan comment, maka baris ini akan diabaikan
% oleh compiler LaTeX.
\begin{comment}
\bibliography{daftar-pustaka}
\end{comment}


%!TEX root = ./ai-jst.tex
%-------------------------------------------------------------------------------
%                            BAB V
%               	 KESIMPULAN dan SARAN
%-------------------------------------------------------------------------------

\chapter{KESIMPULAN DAN SARAN}                

\section{Kesimpulan}
  Jaringan syaraf tiruan merupakan salah satu penerapan yang sangat membantu untuk memecahkan masalah dan persoalan \emph{non-linear} didalam kehidupan sehari-hari, sebagai contoh permasalahan yang menjadi fokus penelitian kami, dalam menilai kelayakan tugas akhir mahasiswa, walaupun metode \emph{Backpropagation} tergolong metode yang sangat dasar dalam dunia jaringan syaraf tiruan tapi metode ini tergolong sangat mampu untuk memcahkan permasalahan yang seperti kami lakukan didalam penelitian ini.Lebih jauh kagi kami menyadari keterbatasan kami dalam melakukan penelitian ini, salah satu nya adalah dengan memanfaatkan \emph{framework} jst dari javascript. 

\section{Saran}
Saran yang dapat ditulis didalam laporan ini adalah:
\begin{enumerate}
	\item Untuk penelitian selajutnya diharapkan kami mampu membangun sebuah model jst tanpa ada bantuan dari \emph{framework} dan dengan \emph{dataset} yang memadai.
	\item Metode yang digunakan tidak hanya \emph{backpropagation} saja, mungkin juga bisa mengunakan metode LSTM untuk data-data yang bersifat runtut waktu. 
\end{enumerate}
% Baris ini digunakan untuk membantu dalam melakukan sitasi
% Karena diapit dengan comment, maka baris ini akan diabaikan
% oleh compiler LaTeX.
\begin{comment}
\bibliography{daftar-pustaka}
\end{comment}


%-----------------------------------------------------------------
%Disini akhir masukan Bab
%-----------------------------------------------------------------


%-----------------------------------------------------------------
% Disini awal masukan untuk Daftar Pustaka
% - Daftar pustaka diambil dari file .bib yang ada pada folder ini
%   juga.
% - Untuk memudahkan dalam memanajemen dan menggenerate file .bib
%   gunakan reference manager seperti Mendeley, Zotero, EndNote,
%   dll.
%-----------------------------------------------------------------
\bibliography{IEEEabrv,daftar-pustaka}
\addcontentsline{toc}{chapter}{DAFTAR PUSTAKA}
%-----------------------------------------------------------------
%Disini akhir masukan Daftar Pustaka
%-----------------------------------------------------------------

\end{document}